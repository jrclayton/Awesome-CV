%%% WORK EXPERIENCE %%%
\cvsection{Work Experience}

\begin{cventries}

    \cventry
    	{Post-doctoral Researcher}
    	{\href{http://www-ibmc.u-strasbg.fr/}{Universit\'{e} de Strasbourg \newline Institut de Biologie Mol\'{e}culaire et Cellulaire}}
    	{Strasbourg, France}
    	{2011-2015}
    	{As a fellow with the French \emph{Institut national de la sant\'{e} et de la recherche m\'{e}dicale} \acr{(INSERM)}, I successfully developed reagents to silence single alleles of polymorphic genes in \emph{An. gambiae} to determine the contribution of each allele to susceptibility to \emph{Plasmodium} infection in the mosquito. My other main research project involved the characterization of a transgenic strain of \emph{An. gambiae} that produces small piwi-like \acr{RNAs} derived from its integrated transgene locus, causing them to become hyper-susceptible to \emph{Plasmodium} infection.}
    
    \cventry
    	{Pre-doctoral Researcher}
    	{\href{http://www.jhsph.edu/departments/w-harry-feinstone-department-of-molecular-microbiology-and-immunology/}{Johns Hopkins Bloomberg School of Public Health \newline Department of Molecular Microbiology \& Immunology}}
    	{Baltimore, \acr{MD}}
    	{2003-2010}
    	{My dissertation research largely involved the characterization of and acquired, sequence-specific anti-viral resistance phenotype observed in \emph{Aedes aegypti} during Alphavirus infection. I also developed a recombinant, mosquitocidal Alphavirus expressing the pro-apoptotic gene \emph{reaper} and evaluated its suitability as a potential biocontrol agent. Another project involved comparative analysis of the Bunyavirus virulence factor \acr{NSs} in mice and mosquitoes.}
    
    \cventry
    	{Post-baccalaureate Researcher}
    	{\href{http://www.embl.de/}{European Molecular Biology Laboratory \newline Kafatos Group}}
    	{Heidelberg, Germany}
    	{2002-2003}
    	{While working in the Kafatos lab I developed a transgenic strain of \emph{An. gambiae} expressing a fluorescent reporter specifically in midgut epithelial cells that had been invaded by the malaria parasite. In parallel, I performed functional analysis of immunity genes and identified a role for the mosquito \acr{NF-$\kappa$B}-like transcription factor \emph{\acr{REL2}} during \emph{Plasmodium} infection.}    

    \cventry
    	{Emerging Infectious Diseases Fellow}
    	{\href{http://www.cdc.gov/}{Centers for Disease Control \& Prevention \newline Entomology Branch}}
		{Atlanta, \acr{GA}}
		{2000-2002}
    	{While an \acr{EID} fellow at the \acr{CDC}, I invented the breakthrough transformation method for \emph{An. gambiae} still in use today. I also characterized the transgenic strains we obtained while working with both \emph{P. vivax} and \emph{P. falciparum} malaria parasites in a \emph{BSL-2} environment. Field work opportunities included surveillance for West Nile Virus and assisting in \acr{CDC's} emergency response during the 2001 Anthrax bioterror crisis.}

	\cventry
		{Research Assistant}
		{\href{http://ib.berkeley.edu/}{University of California \newline Department of Integrative Biology}}
		{Berkeley, \acr{CA}}
		{1998-1999}
		{While in the Brent Mishler's laboratory I evaluated the utility of \acr{RNA} secondary structures to reconstruct deep phylogenies of green plants.}

\end{cventries}